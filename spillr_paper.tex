\documentclass{bioinfo}
\copyrightyear{2024} \pubyear{2024}



% tightlist command for lists without linebreak
\providecommand{\tightlist}{%
  \setlength{\itemsep}{0pt}\setlength{\parskip}{0pt}}



\usepackage{euflag}
\usepackage{booktabs}
\usepackage{longtable}
\usepackage{array}
\usepackage{multirow}
\usepackage{wrapfig}
\usepackage{float}
\usepackage{colortbl}
\usepackage{pdflscape}
\usepackage{tabu}
\usepackage{threeparttable}
\usepackage{threeparttablex}
\usepackage[normalem]{ulem}
\usepackage{makecell}
\usepackage{xcolor}

% hyperref makes the margins screwy.
% https://groups.google.com/forum/#!topic/latexusersgroup/4W_SwGk6zx4
% http://ansuz.sooke.bc.ca/software/latex-tricks.php
% \usepackage[colorlinks=true, allcolors=blue]{hyperref}

\access{Advance Access Publication Date:   }
\appnotes{Original Paper}

\begin{document}


\firstpage{1}

\subtitle{Systems biology}

\title[\texttt{Spillover\ Compensation}]{\texttt{spillR}: Spillover
Compensation in Mass Cytometry Data}

\author[FirstAuthorLastName \textit{et~al}.]{
Marco Guazzini\,\textsuperscript{1}, Alexander G.
Reisach\,\textsuperscript{2}, Sebastian
Weichwald\,\textsuperscript{3}, Christof Seiler\,\textsuperscript{1,4,5}
}

\address{
\textsuperscript{1}Department of Advanced Computing Sciences, Maastricht
University, The Netherlands\\
\textsuperscript{2}Université Paris Cité, CNRS, MAP5, F-75006 Paris,
France\\
\textsuperscript{3}Department of Mathematical Sciences, University of
Copenhagen, Denmark\\
\textsuperscript{4}Mathematics Centre Maastricht, Maastricht University,
The Netherlands\\
\textsuperscript{5}Center of Experimental Rheumatology, Department of
Rheumatology, University Hospital Zurich, University of Zurich,
Switzerland\\
}

\corresp{To whom correspondence should be addressed. E-mail:
christof.seiler@maastrichtuniversity.nl}

\history{Received on XXX; revised on XXX; accepted on XXX}

\editor{Associate Editor: XXX}

\abstract{
\textbf{Motivation:} Channel interference in mass cytometry can cause
spillover and may result in miscounting of protein markers.
\citet{catalyst} introduce an experimental and computational procedure
to estimate and compensate for spillover implemented in their R package
\texttt{CATALYST}. They assume spillover can be described by a spillover
matrix that encodes the ratio between the signal in the unstained
spillover receiving and stained spillover emitting channel. They
estimate the spillover matrix from experiments with beads. We propose to
skip the matrix estimation step and work directly with the full bead
distributions. We develop a nonparametric finite mixture model and use
the mixture components to estimate the probability of spillover.
Spillover correction is often a pre-processing step followed by
downstream analyses, and choosing a flexible model reduces the chance of
introducing biases that can propagate downstream.\\
\textbf{Results:} We implement our method in an R package
\texttt{spillR} using expectation-maximization to fit the mixture model.
We test our method on simulated, semi-simulated, and real data from
\texttt{CATALYST}. We find that our method compensates low counts
accurately, does not introduce negative counts, avoids overcompensating
high counts, and preserves correlations between markers that may be
biologically meaningful.\\
\textbf{Availability:} Our new R package \texttt{spillR} is on
Bioconductor at bioconductor.org/packages/spillR. All experiments and
plots can be reproduced by compiling the R markdown file
\texttt{spillR\_paper.Rmd} at github.com/ChristofSeiler/spillR\_paper.\\
\textbf{Contact:}christof.seiler@maastrichtuniversity.nl\\
\textbf{Supplementary information:} Supplementary data are available at
Bioinformatics Online.}

\maketitle

\section{Introduction}

Mass cytometry makes it possible to count a large number of proteins
simultaneously on individual cells
\citep{bandura2009mass, bendall2011single}. Although mass cytometry has
less spillover---measurements from one channel overlap less with those
of another---than flow cytometry \citep{sp-c, novo2013generalized},
spillover is still a problem and affects downstream analyses such as
differential testing \citep{diffcyt, seiler2021cytoglmm} or
dimensionality reduction \citep{scater}. Reducing spillover by careful
design of experiment is possible \citep{takahashi2017mass}, but a purely
experimental approach may be neither efficient nor sufficient
\citep{lun2017influence}.

\citet{catalyst} propose a method for addressing spillover by conducting
an experiment on beads. This experiment measures spillover by staining
each bead with a single type of antibody. The slope of the regression
line between target antibodies and non-target antibodies represents the
spillover proportion between channels. \citet{miao2021ab} attempt to
solve spillover by fitting a mixture model. Our contribution combines
the solutions of \citet{catalyst} and \citet{miao2021ab}. We still
require a bead experiment, as in \citet{catalyst}, but estimate
spillover leveraging a statistical model, as in \citet{miao2021ab}. Both
previous works rely on an estimate for the spillover matrix, which
encodes the pairwise spillover proportion between channels. We avoid
estimating a spillover matrix and instead model spillover by fitting a
mixture model to the counts observed in the bead experiment. Our main
new assumption is that the spillover distribution---not just the
spillover proportion---from the bead experiment carries over to the
biological experiment. In other words, we transfer the spillover
distribution to the real experiment instead of just the spillover
proportion encoded in the spillover matrix.

In Section \ref{methods}, we present our mixture model and link it to
calculating spillover probabilities for specific count values. Our
estimation procedure is based on an EM algorithm, and implemented in our
new R package \texttt{spillR}. In Section \ref{results}, we conduct
experiments on simulated, semi-simulated, and real data obtained from
the \texttt{CATALYST} R package \citep{catalyst}. Section
\ref{discussion} discusses our experiments and relates our findings to
\texttt{CATALYST}.

\section{Methods}

\label{methods}

In this section we first illustrate our method \texttt{spillR} (as well
as a simple baseline \texttt{spillR-naive}) by an example, and then
describe the algorithm and its underlying assumptions. Regarding
terminology, mass cytometry counts are often referred to as dual counts
or signal intensity; we refer to them as counts to emphasize their
nature as non-negative integers, as opposed to possibly real-valued
intensities.

\subsection{Example}

\begin{figure}

{\centering \includegraphics[width=0.7\linewidth]{spillr_paper_files/figure-latex/method-example-1} 

}

\caption{Panel A shows a density plot of target and spillover markers based on the beads experiment, Panel B shows spillover probability for Yb173Di estimated by \texttt{spillR}, and Panel C compares spillover compensation on real cells by our methods and \texttt{CATALYST}. Counts are arcsinh transformed with cofactor of five \citep{bendall2011single}, zero counts are not shown. As seen in Panel C, our baseline method \texttt{spillR-naive} performs similarly to \texttt{CATALYST} and compensates the first peak of the uncorrected data (red) between about 2 and 4 as spillover. By contrast, \texttt{spillR} is sensitive to the difference in shape between the peaks in the bead data (Panel A) and the first peak in the real data (Panel C red), and only compensates the part of the red curve as spillover that matches the bead experiment. This figure is an example of the diagnostic plot obtained when using the function \texttt{plotDiagnostics} in \texttt{spillR}.}\label{fig:method-example}
\end{figure}

Figure \ref{fig:method-example} illustrates our procedure using a
dataset from the \texttt{CATALYST} package as an example. There are four
markers, HLA-DR (Yb171Di), HLA-ABC (Yb172Di), CD8 (Yb174Di), and CD45
(Yb176Di), that spill over into the target marker, CD3 (Yb173Di). The
markers have two names: the first name is the protein name and the
second name in brackets is the conjugated metal. There are bead
experiments for each of the spillover markers.

Panel A depicts the marker distributions from the beads experiment. We
see that for this marker the bead experiments are high-quality as the
target marker Yb173Di is concentrated around six, similarly to the
experiment with real cells. This suggests that the spillover marker
values can be transferred to the real experiments. Marker Yb172Di shows
large spillover into Yb173Di, and suggests that the left tail of the
first mode of the distribution observed on real cells may be attributed
to that marker. The other spillover markers have low counts in the bead
experiment, making it justifiable to set some or all of the low counts
on real cells to zero.

Panel B shows a curve representing our spillover probability estimates.
We can see that the probability of spillover is high for counts with a
high density of spillover markers (panel A) and a low density of the
target marker (panel C). If the spillover probability is close to one,
our correction step assigns most cells to spillover. Counts above \(4\)
stem from spillover with probability zero (and from the actual target
with probability one), which means that our procedure keeps them at
their raw uncorrected value.

Panel C displays the distribution of our target marker, CD3 (Yb173Di)
before and after spillover correction. We observe few real counts (red)
below \(2\), so although all methods perform strong compensation in this
range, there is little visible difference between uncompensated and
compensated counts. Above \(2\) there is a clear distinction between the
compensation methods. \texttt{CATALYST}, like our baseline
\texttt{spillR-naive}, compensates nearly all counts forming the first
peak of the raw counts (red, between \(2\) and \(4\)) as spillover. By
contrast, \texttt{spillR} compensates only where the density of
spillover markers in the bead experiment shown in panel A is high
(e.g.~Yb172Di spillover peaks at around \(2.7\)). As a result, it does
not compensate for all of the counts forming the first peak of the red
curve and compensates more counts between \(2\) and \(3\) than between
\(3\) and \(4\). While \texttt{CATALYST} shifts large counts (around
\(6\)) slightly to the left, our methods leave them unchanged as the
bead experiment shows no spillover in this range. Our baseline method
\texttt{spillR-naive} is similar to \texttt{CATALYST} in the low and
medium range, but keeps higher counts unchanged.

\subsection{Definition of Spillover Probability and Assumptions}

We observe a count \(Y_i\) of a target marker in cell \(i\). We model
the observed \(Y_i\) as a finite mixture \citep{mclachlan2019finite} of
unobserved true marker counts \(Y_i \mid Z_i = 1\) and spillover marker
counts \(Y_i \mid Z_i = 2, \dots, Y_i \mid Z_i = K\) with mixing
probabilities \(\pi_{k} = P(Z_i = k)\) for \(k = 1, \dots, K\), \[
P(Y_i = y) = \sum_{k = 1}^K \pi_k \, P(Y_i = y \mid Z_i = k).
\] The mixing probability \(\pi_1\) is the proportion of true signal in
the observed counts. The other \(K-1\) mixing probabilities are the
proportions of spillover. The total sum of mixing probabilities equals
one, \(\sum_k \pi_k = 1\). The total number of markers in mass cytometry
panels is between 30 and 40 \citep{bendall2011single}, but only a small
subset of three to four markers spill over into the target marker
\citep{catalyst}. So, typically \(K = 1+3\) or \(K = 1+4\).

Experimentally, we only measure samples from the distribution of
\(Y_i\). The probabilities \(\pi_k\) and true distributions
\(P(Y_i = y \mid Z_i = k)\) are unobserved, and we need to estimate them
from data. In many applications, the mixture components are modeled to
be in a parametric family, for example, the negative binomial
distribution. As spillover correction is a pre-processing step followed
by downstream analyses, choosing the wrong model can introduce biases in
the next analysis step. To mitigate such biases, we propose to fit
nonparametric mixture components. We make two assumptions that render
the components and mixture probabilities identifiable:

\begin{itemize}
\item
  (A1) Spillover distributions are the same in bead and real
  experiments.

  The distribution of \(Y_i \mid Z_i = k\) for all \(k > 1\) is the same
  in beads and real cells. This assumption allows us to learn the
  spillover distributions of \(Y_i \mid Z_i = k\) for all \(k > 1\) from
  experiments with beads, and transfer them to the experiment with real
  cells. This assumption relies on high-quality single-stained bead
  experiments that measure spillover in the same range as the target
  biological experiment. In other words, a bead experiment for our
  method works best if the distribution of bead cells is similar to the
  distribution of real cells.
\item
  (A2) For each cell \(i\), the observed count \(Y_i\) can only be due
  to one marker.

  This assumption is already implied by the statement of the mixture
  model. It allows us to calculate the spillover probability for a given
  count \(Y_i = y\) from the posterior probability that it arises
  through spillover from markers \(k > 1\), \[
  \begin{split}
  P(\text{spillover} \mid Y_i = y) & = P(Z_i > 1 \mid Y_i = y) \\
  & = 1 - P(Z_i = 1 \mid Y_i = y) \\
  & = 1 - \frac{\pi_1 \, P(Y_i = y \mid Z_i = 1)}{P(Y_i = y)}.
  \end{split}
  \] To parse this calculation, recall that in mixture models the
  \(\pi_1\) is the prior probability, \(P(Y_i = y \mid Z_i = 1)\) is the
  conditional probability given the mixture component, and the
  denominator \(P(Y_i = y)\) is the marginal distribution. Applying
  Bayes rule leads to the posterior probability.
\end{itemize}

\subsection{Estimation of Spillover Probability}

We propose a two-step procedure for estimating the spillover
probability. In step 1, we estimate mixture components and mixture
probabilities. We refine these estimates using the EM algorithm
\citep{dempster1977maximum}. In step 2, we use these probability
estimates to assign counts to spillover or target marker signal.

We denote the \(n \times K\) count matrix as \(\mathbf{Y} = (y_{ik})\)
with real cells in the first column and beads in columns two and higher.
To simplify mathematical notation but without loss of generality, we
assume that the number of events from real and bead experiments have the
same \(n\). In practice, the number of events from bead experiments is
much smaller than from real experiments. For \(k > 1\), the \(k\)th
column of \(\mathbf{Y}\) contains marker counts for a given spillover
marker, which represents the empirical spillover distribution of marker
\(k\) into the target marker represented by the first column of
\(\mathbf{Y}\).

\subsubsection{EM Algorithm}

\begin{itemize}
\item
  Initialization: For the mixture probability vector, we assign
  probability \(0.9\) to the the target marker and divide the
  probability \(0.1\) among the spillover markers, \[
  \hat{\pi}_{1} = 0.9 \text{ and } \hat{\pi}_i = 0.1/(K-1) \text{ for all } i > 1.
  \] The procedure is not sensitive to the choice of the initial mixture
  probability vector and other initializations are possible but may be
  slower to converge. Then, we initialize the \(k\)th mixture component
  using its probability mass function (PMF) after smoothing and
  normalizing, \(\widehat{P}(Y_i = y \mid Z_i = k)\). We smooth the PMF
  using kernel density estimation implemented in the R function
  \texttt{density} with the default option for selecting the bandwith of
  a Gaussian kernel.
\item
  E-step: We evaluate the posterior probability of a count \(y\)
  belonging to component \(k\) (that is, originating from marker \(k\)),
  \[
  \widehat{P}\left(Z_i = k \mid Y_i = y \right) = 
  \frac
  { \hat{\pi}_k \, \widehat{P}(Y_i = y \mid Z_i = k) }
  { \sum_{k' = 1}^K \hat{\pi}_{k'} \, \widehat{P}(Y_i = y \mid Z_i = k') }.
  \]
\item
  M-step: We estimate the new mixture probability vector from posterior
  probabilities, \[
  \hat{\pi}_k = 
  \frac{1}{n} \sum_{i = 1}^n \widehat{P}\left(Z_i = k \mid Y_i = y \right),
  \] and estimate the new target marker distribution by smoothing and
  normalizing. Here, we use the R function \texttt{density} again,
  weighing each observation according to its posterior probability
  \(\widehat{P} \left(Z_i = 1 \mid Y_i = y \right)\). We only update the
  target marker distribution, \(\widehat{P}(Y_i = y \mid Z_i = 1)\), and
  keep the other bead distributions,
  \(\widehat{P}(Y_i = y \mid Z_i = k)\) for all \(k > 1\), fixed at
  their initial value.
\end{itemize}

To refine our estimates, we iterate over the E and M-steps until
estimates stabilize. We stop iterating when \(\hat{\pi}_1\) changes less
than \(10^{-5}\) from the previous iteration. The final output is the
spillover probability curve with estimates at discrete points in the
support of \(Y_i\), \[
\widehat{P}(\text{spillover} \mid Y_i = y) = 1 - \widehat{P}(Z_i = 1 \mid Y_i = y).
\]

We rely on assumption (A1) to justify updating only the distribution of
the target marker. We rely on assumption (A2) to justify calculating the
spillover probability from the mixture model. We refer to the
supplementary data for a step-by-step example of our EM algorithm.

\subsubsection{Spillover Decision}

To perform spillover compensation, we draw from a Bernoulli distribution
with the spillover probability as parameter to decide whether or not to
assign a given count to spillover. We mark counts attributed to
spillover by setting them to a user-specified value. We recommend a
value of zero to maintain the overall cellular composition of the
sample, or a value such as \texttt{NA} or \(-1\) to mark spillover
counts for separate treatment in downstream analyses (for example,
calculating means only over non-spillover counts).

\subsection{Baseline Method}

We compare our mixture method to a naive baseline method
\texttt{spillR-naive} that considers only the bead distributions. We
replace the real cells in the first component \(k = 1\) with their bead
distribution. Similarly to our standard \texttt{spillR} method, we
estimate the bead PMF of each bead \(k\) with the kernel density
estimator \texttt{density}, \(\widehat{P}(Y_i = y \mid Z_i = k)\). Then,
for all count values \(y\) in the range of the bead counts, we
separately normalize the PMF at each value \(Y_i = y\) and calculate the
spillover probability as \[
\widehat{P}(\text{spillover} \mid Y_i = y) = 1 - \frac{\widehat{P}(Y_i = y \mid Z_i = 1)}{\sum_{k' = 1}^K \widehat{P}(Y_i = y \mid Z_i = k')}.
\] We proceed as in our standard \texttt{spillR} method to decide
whether or not to assign a given count to spillover. This is a
computationally efficient and simple baseline that assigns counts to
markers in proportion to their density at that count value in the
corresponding bead experiment.

\section{Results}

\label{results}

We first evaluate our new method \texttt{spillR} on simulated datasets.
We probe our method to experimentally find its shortcomings. Then, we
compare \texttt{spillR} to the non-negative least squares method
implemented in the R package \texttt{CATALYST} on real and
semi-simulated data from the same package.

\subsection{Simulated Data}

\label{simulated-data}

\begin{figure}

{\centering \includegraphics[width=0.9\linewidth]{spillr_paper_files/figure-latex/simulated-experiments-plot-1} 

}

\caption{Three experiments testing our assumptions and sensitivity to bimodal bead distribution. For each experiment the top row are mean values over the entire range of the experimental setups. The mean values for \texttt{spillR} are computed on values not marked as \texttt{NA}, so the mean ignores the counts attributed to spillover. The bottom row are density plots for three parameter settings to illustrate the generated distributions. $Y$ is the distribution with spillover. $Y \mid Z = 1$ is the distribution without spillover. $Y \mid Z = 2$ is the spillover. mean($Y$) is the average of the distribution with spillover. mean($Y \mid Z = 1$) is the average count without spillover. \texttt{spillR} mean($Y$) is the average count after correcting $Y$.}\label{fig:simulated-experiments-plot}
\end{figure}

We choose three different experiments to test \texttt{spillR} under
different bead and real cell distributions. We explore a wide range of
possible parameter settings. Figure \ref{fig:simulated-experiments-plot}
has three panels, each representing one experimental setup. The first
two panels test our assumptions (A1) and (A2). The third panel tests the
sensitivity of \texttt{spillR} to bimodal bead distributions. For all
three experiments, we model counts using a Poisson distribution with
parameter \(\lambda\). We simulate 10,000 real cells with
\(\lambda = 200\), and 1,000 beads with \(\lambda = 70\), and a
spillover probability of \(0.5\). Unless otherwise specified, the bead
data are drawn from the true spillover distribution. The other
parameters and statistical dependencies are specific to each experiment.
The details of the generative models are given in the supplementary
data. We repeat each simulation 20 times and report averages over the 20
replications.

Each panel of Figure \ref{fig:simulated-experiments-plot} has two rows
of plots. The plot in the first row represents the summary of the means
for each experimental setup as a function of their respective parameter
\(\tau\). This parameter has a different meaning in each setup. To
visualize the different experiments, we summarize the full distributions
with the true simulated signal mean (black), the uncorrected mean
(orange), and the \texttt{spillR} corrected mean (green). Plots on the
second row illustrate the simulated data distributions for three
selected parameters \(\tau\) picked from the experimental setup. The
yellow density curve shows the observed counts \(Y\). The black density
curve shows the distribution of target cell counts. The blue density
curve shows the distribution of spillover counts. The yellow density
curve represents the data \(Y\) we would observe in practice. We
simulate this data using the models in the supplementary data. The goal
of the experiment is to estimate the mean of the true counts (black
density curve) as accurately as possible from the observed counts
(yellow density curve). Using \texttt{NA} imputation for spillover
counts, the average of the compensated observed counts when ignoring
\texttt{NA} values is equal to the mean of the true counts if all
spillover counts are correctly identified as such.

In the first experiment (panel A), we shift the spillover in the beads
experiment away from the true spillover to probe (A1). We test a range
of bead shifts from no shift at \(\tau = 0\) to \(\tau = -10\). At
\(\tau = -10\), the measured spillover (the first mode of the yellow
density) is shifted away from the actual spillover (the blue density),
causing both the observed and compensated mean to be lower than the true
mean. This may be the case in a low-quality bead experiment. As \(\tau\)
gets closer to zero, the first mode of the yellow density moves towards
the blue density (as may be the case in a higher quality bead
experiment), and the compensated signal moves closer to the true mean.

In the second experiment (panel B), we mix target and spillover to
explore the robustness of our method with respect to our second
assumption (A2). One way to think about this is that the mixture is a
form of model misspecification. Our mixture model is undercomplete,
which means that there are more true mixture components than we observe
in the beads experiment. If \(\tau = 0\), then assumption (A2) is
correct, but for \(\tau = 0.5\) the assumption (A2) is maximally
violated. The true mean decreases with increasing \(\tau\).
\texttt{spillR} compensates well as long as \(\tau\) is close to zero,
but does not adapt immediately as the spillover distribution gets closer
to the target distribution for increasing \(\tau\). When the mean of the
spillover distribution crosses the mid-way point to that of the target
marker distribution at \(\tau \approx 0.25\), the mean of counts
compensated by \texttt{spillR} flips to the mean of the observed data,
until at \(\tau = 0.5\) all three distributions and their means are the
same.

In the third experiment (panel C), we model spillover with a bimodal
distribution. Here \(\tau\) is the mixing probability of the two modes.
The locations of the two spillover modes are fixed. If \(\tau = 0\) or
\(\tau = 1\), then spillover is unimodal. If \(\tau = 0.5\), the first
mode of the bimodal bead distribution is left to the signal mode, and
the second mode is to the right. The corrected mean is closer to the
true mean than the uncorrected mean across the test range.

\subsection{Real Data}

\label{real-data}

\begin{figure}

{\centering \includegraphics[width=0.95\linewidth]{spillr_paper_files/figure-latex/spillr-vignette-1} 

}

\caption{Comparison of compensation methods and uncorrected counts on real data. Counts are arcsinh transformed with cofactor of five \citep{bendall2011single}.}\label{fig:spillr-vignette}
\end{figure}

We compare our methods to \texttt{CATALYST} on one of the example
datasets in the \texttt{CATALYST} package. The dataset consists of an
experiment with real cells and corresponding single-stained bead
experiments. The experiment on real cells has 5,000 peripheral blood
mononuclear cells from healthy donors measured on 39 channels. The
experiment on beads has 10,000 cells measured on 36 channels with the
number of beads per metal label ranging from 112 to 241.

In Figure \ref{fig:spillr-vignette}, we show the comparison of our
methods to \texttt{CATALYST} on the same markers as their original paper
\citep{catalyst} in their Figure 3B. In the original experiment, they
conjugate the three proteins CD3, CD8, and HLA-DR with two different
metal labels. For example, they conjugate CD8 with Yb174Di (Yb is the
metal and the number indicates the number of nucleons of the isotope)
and La139Di. As in their plot, our columns correspond to the different
metal labels. In the first column are isotopes that are close to one
another, which can cause spillover. In the second column are isotopes
that are far from one another, which should not cause spillover. Each
row represents a comparison for a target channel (on the horizontal
axis) and the potential spillover channel (on the vertical axis). We
visualize the joint distributions using two-dimensional histograms.

In all six panels (A--F), we observe that \texttt{spillR} compensates
most strongly in the low counts, whereas \texttt{CATALYST} compensates
strongly in the middle range. From the experimental setup, we expect
strong spillover in the first column, and little spillover in the second
column. In the first column, \texttt{spillR} seems to undercompensate in
panels A and C, but compensates strongly in panel E, whereas we observe
the opposite trend for \texttt{CATALYST}. In the second column, we find
that \texttt{CATALYST} compensates substantially in panel B, and
\texttt{spillR} compensates substantially in panels D and F, even though
we expect little spillover. The baseline method \texttt{spillR-naive}
can be seen to compensate strongly in all cases.

A characteristic pattern can be seen in panel C, CD3 (Yb173Di) against
HLA-ABC (Yb172Di). \texttt{CATALYST} compensates strongly in the middle
range and removes the spherical pattern that shows correlation between
the two markers. \texttt{spillR} preserves this correlation structure
and only compensates the lower counts of CD3 (Yb173Di). This highlights
a key difference between \texttt{spillR} and \texttt{CATALYST}:
\texttt{spillR} identifies counts that may arise from spillover and
replaces them with a user-specified value (e.g.~0, \texttt{NA}, or -1),
whereas \texttt{CATALYST} shrinks counts across the entire range to
compensate for spillover.

The color code of the two-dimensional histograms indicates the absolute
number of cells that fall into one hexagon bin. The uncorrected and
\texttt{spillR} corrected histograms can contain different absolute
numbers of cells, even for identical distributions. This is due to a
rounding step in \texttt{spillR} that converts raw counts to integers.
Raw mass cytometry data may not be true count data because the
proprietary post-processing of the manufacturer often performs a
randomization step when exporting the data. The uncorrected counts do
not undergo this pre-processing step, and \texttt{CATALYST} does not
perform this pre-processing step either. This also explains the
different patterns in panel B. \texttt{spillR} has horizontal stripes
that correspond to non-integer values not in the support of the
distribution for \texttt{spillR}. We leave the decision to apply
re-randomization of the count data for downstream analysis up to the
user. Our rationale is that the user should see the differences in this
pre-processing step and how it propagates to the results.

The average computation time for the experiment shown in Figure
\ref{fig:spillr-vignette} with 100 replications on an Apple M1 with 8
cores and 16 GB of RAM is \(10.6\) seconds for \texttt{spillR}, \(0.43\)
seconds for \texttt{CATALYST}, and \(0.45\) seconds for
\texttt{spillR-naive}. The computational costs scale linearly in the
number of cells and number of spillover markers. This allows for
processing large-scale datasets.

\subsection{Semi-Simulated Data}

\label{semi-simulated-data}

\begin{figure}

{\centering \includegraphics[width=1\linewidth]{spillr_paper_files/figure-latex/semi-simulated-plot-1} 

}

\caption{Comparison of compensation methods and uncorrected counts on semi-synthetic data (\texttt{spillR} and \texttt{spillR-naive} are set to impute spillover values with $0$). The vertical dashed line helps to interpret the spillover correction. It indicates the original mode of the bead distribution of Yb172Di at $2.7$, before overwriting it with the first peak of the real observations of Yb173Di. Counts are arcsinh transformed with cofactor of five \citep{bendall2011single}. The zero percentages are averages over all three experiments.}\label{fig:semi-simulated-plot}
\end{figure}

We compare \texttt{spillR} and \texttt{CATALYST} on semi-simulated data
in order to elucidate differences between \texttt{spillR} and
\texttt{CATALYST}, and to evaluate the performance of \texttt{spillR}
when more than one marker spills into the target marker. We create
semi-simulated datasets by overwriting the bead distribution for the
target marker CD3 (Yb173Di). We take the first mode of the count
distribution of CD3 (Yb173Di) observed in real cells (the counts from
\(1.44\) to \(4.79\) on the transformed scale) as a reference range. We
overwrite the bead distribution of Yb172Di, which dominates this range,
by the observed cell distribution in the same range with three different
shift values: no shift is \(0\), subtracting \(0.47\) on the transformed
scale, and subtracting \(0.94\) on the transformed scale. We further
subsample without replacement from this new bead distribution to keep
the same number of beads as in the original dataset. Figure
\ref{fig:semi-simulated-plot} shows the three different beads experiment
datasets in row A and the resulting compensations in row B.

In the first column of Figure \ref{fig:semi-simulated-plot}, the bead
distributions are equal to the original dataset from Figure
\ref{fig:method-example} except Yb172Di is now perfectly aligned with
the first mode of the distribution of real cells (red curve in row B).
In the second and third column, we shift the bead distribution of
Yb172Di by \(0.47\) and \(0.94\). All three methods correctly compensate
the spillover mode when no shift is present (first column).
\texttt{CATALYST} and \texttt{spillR-naive} compensates more
aggressively in the medium shift cases (second column), while
\texttt{spillR} is more moderate and compensates only the left hand tail
of the spillover mode. For a shift of \(0.94\) (third column), the three
methods differ: \texttt{CATALYST} shrinks counts towards zero, shifting
the entire spillover towards zero (resulting in many counts between
about 1 and 3), \texttt{spillR} compensates lightly on the left hand
tail, and \texttt{spillR-naive} compensates aggressively leaving only a
small right hand tail. This experiment illustrates how \texttt{spillR}
compensates most strongly for counts that can be attributed to spillover
following the distribution observed in the beads experiment.

\section{Discussion}

\label{discussion}

The sensitivity analysis in Section \ref{simulated-data} illustrates the
performance of \texttt{spillR} under different conditions. The
experiment for (A1) shows that the mean count after \texttt{spillR}
correction is closer to the true mean over a wide range of bead shifts.
This indicates that our method can perform well even if the bead
experiments are imperfect. If the difference between distributions of
beads and real cells is large, then one option is to rerun the bead
experiments to reduce this gap. The experiment for (A2) shows that our
method is robust to model misspecification. Additionally,
misspecification can be addressed by adding all channels if necessary.
The increase in computational cost when adding channels is relatively
minor as our method scales linearly in the number of spillover markers.
The experiment on bimodal bead distributions shows that the mean count
after correction is still closer to the true mean even with bimodal bead
distributions and even if the spillover is larger than the true signal.

In our comparison with \texttt{CATALYST} on real data (Section
\ref{real-data}) and semi-simulated data (Section
\ref{semi-simulated-data}), we observe the effect of the two different
correction strategies. \texttt{CATALYST} shrinks all counts towards zero
by minimizing a non-negative least squares objective. It assumes that
spillover is linear up to counts of 5,000. The applied shrinkage is the
same for low counts (e.g., below 10) and high counts (e.g., more than
100). By contrast, \texttt{spillR} does not require linearity of the
spillover, but assumes that the distribution on the beads experiment
carries over to the real cells experiment. If counts are in the
spillover range (which mostly applies to low counts), they are corrected
strongly and set to a user-specified imputation value. If counts are not
in the spillover range, they are left unchanged. Among the unchanged
counts, correlations between markers are preserved. The marker
correlation between HLA-ABC (Yb172Di) and CD3 (Yb173Di) shown in the
first column of Figure \ref{fig:spillr-vignette} illustrates this point.
\texttt{CATALYST} removes the positively correlated count concentration,
whereas \texttt{spillR} keeps it. Compensation methods have to balance
between compensating for spillover while keeping potentially
biologically meaningful signals for unbiased downstream analyses. In
this example, further experiments on the correlation structure between
these markers would be necessary to resolve the discrepancy between the
two methods. This is an important point as discovering correlations
between markers can lead to the discovery of new clusters or signaling
networks.

Our baseline method, \texttt{spillR-naive}, illustrates the behavior of
more aggressive compensation by considering only the bead distribution.
If our baseline method compensates aggressively in a certain range, this
is because most bead counts observed in that range are spillover counts.
This approach highlights the allure and pitfalls of overcorrecting.
While in Figure \ref{fig:method-example} it may seem that
\texttt{spillR-naive} compensates for all spillover by setting the first
mode to zero (just like \texttt{CATALYST}), a closer inspection reveals
that discrepancies between the bead spillover distribution and the first
mode of the real cell distribution are not taken into account by either
method, but do reflect in the compensation of \texttt{spillR}. A similar
pattern can be seen in panels A, B, and C of Figure
\ref{fig:spillr-vignette}. This behavior reflects our assumption (A1)
and highlights the role of bead experiments in the compensation process
performed by \texttt{spillR}.

In the cell and bead data from \texttt{CATALYST}, our assumption (A1)
seems to hold in some cases, but is violated in others. In our leading
example in Figure \ref{fig:method-example} we can see that the mode of
the bead distribution of marker Yb173Di is very close to the mode in the
cell data, supporting our assumption. In other cases however, the target
marker mode differs more strongly between cell and bead data. We
therefore recommend testing our assumptions against the available data
and background knowledge. Comparing the target marker mode on cells and
beads may serve as a first test for assumption (A1), and background
knowledge on the markers may aid in assessing the plausibility of
assumption (A2).

To understand and assess its applicability and performance,
\texttt{spillR} offers a diagnostic plot (Figure
\ref{fig:method-example}) of the spillover probability curve. We can
judge if the curve makes sense by comparing it to the observed count and
bead distributions. Methods based on non-negative least squares such as
\texttt{CATALYST} are harder to diagnose as they minimize a cost
function with no clear biological interpretation. In our view, one
strength of \texttt{spillR} is that it does not assume a specific
parametric model for count data. We believe that this is crucial because
spillover compensation precedes many downstream analysis steps, and
avoiding the introduction of bias is thus our priority.

In our experiments, we observe that the different methods may over- or
undercompensate in different cases. The original experiment shown in
Figure \ref{fig:spillr-vignette} is designed to produce considerable
spillover in the settings of the first column, and little spillover in
the second column. Nonetheless, we observe weak and strong compensation
by both \texttt{spillR} and \texttt{CATALYST} in both columns. Our
diagnostic plots show overlap between the spillover marker distributions
on beads and the first mode of the real cell distribution, thus the
compensation by \texttt{spillR} is consistent with assumption (A1). The
strong compensation performed by \texttt{spillR-naive} in both columns
highlights the difficulty of balancing necessary and excessive
compensation. \texttt{spillR} is designed to err on the side of
preserving potentially meaningful patterns unless the bead distributions
clearly suggest spillover. Overall, our results indicate that the
performance of both \texttt{spillR} and \texttt{CATALYST} rests on the
plausibility of their respective assumptions in the individual case.
Since \texttt{spillR} relies on a different set of assumptions, it
offers a complementary solution. Note that in principle it is also
possible to combine the methods, for example by using \texttt{spillR} to
identify spillover, and \texttt{CATALYST} to compensate for it.

Our basic method can also be applied to imaging mass cytometry
\citep{angelo2014multiplexed, giesen2014highly, bodenmiller2016multiplexed},
although it will likely be beneficial to incorporate a spatial
regularization term that enforces similarity between neighboring
spillover estimates. We consider the design and evaluation of such an
extension of our method a promising direction for future work.

\section*{Acknowledgments}
\addcontentsline{toc}{section}{Acknowledgments}

We thank EuroBioC2022 for awarding Marco Guazzini a travel award to
present a preliminary version of \texttt{spillR} in Heidelberg. We thank
Antoine Chambaz for his feedback on an earlier draft that substantially
improved the paper.

\section*{Funding}
\addcontentsline{toc}{section}{Funding}

This work was supported by the European Union's Horizon 2020 research
and innovation program under a Marie Sk\l{}odowska-Curie grant {[}945332
to A.G.R.{]} \euflag.


% Bibliography
\bibliographystyle{natbib}
\bibliography{bibliography.bib}

\end{document}
